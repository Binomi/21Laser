% Für Bindekorrektur als optionales Argument "BCORfaktormitmaßeinheit", dann
% sieht auch Option "twoside" vernünftig aus
% Näheres zu "scrartcl" bzw. "scrreprt" und "scrbook" siehe KOMA-Skript Doku
\documentclass[12pt,a4paper,titlepage,headinclude,bibtotoc]{scrartcl}


%---- Allgemeine Layout Einstellungen ------------------------------------------

% Für Kopf und Fußzeilen, siehe auch KOMA-Skript Doku
\usepackage[komastyle]{scrpage2}
\pagestyle{scrheadings}
\automark[section]{chapter}
\setheadsepline{0.5pt}[\color{black}]

%keine Einrückung
\parindent0pt

%Einstellungen für Figuren- und Tabellenbeschriftungen
\setkomafont{captionlabel}{\sffamily\bfseries}
\setcapindent{0em}

\usepackage{caption}

%---- Weitere Pakete -----------------------------------------------------------
% Die Pakete sind alle in der TeX Live Distribution enthalten. Wichtige Adressen
% www.ctan.org, www.dante.de

% Sprachunterstützung
\usepackage[ngerman]{babel}

% Benutzung von Umlauten direkt im Text
% entweder "latin1" oder "utf8"
\usepackage[utf8]{inputenc}

% Pakete mit Mathesymbolen und zur Beseitigung von Schwächen der Mathe-Umgebung
\usepackage{latexsym,exscale,amssymb,amsmath}

% Weitere Symbole
\usepackage[nointegrals]{wasysym}
\usepackage{eurosym}

% Anderes Literaturverzeichnisformat
%\usepackage[square,sort&compress]{natbib}

% Für Farbe
\usepackage{color}

% Zur Graphikausgabe
%Beipiel: \includegraphics[width=\textwidth]{grafik.png}
\usepackage{graphicx}

% Text umfließt Graphiken und Tabellen
% Beispiel:
% \begin{wrapfigure}[Zeilenanzahl]{"l" oder "r"}{breite}
%   \centering
%   \includegraphics[width=...]{grafik}
%   \caption{Beschriftung} 
%   \label{fig:grafik}
% \end{wrapfigure}
\usepackage{wrapfig}

% Mehrere Abbildungen nebeneinander
% Beispiel:
% \begin{figure}[htb]
%   \centering
%   \subfigure[Beschriftung 1\label{fig:label1}]
%   {\includegraphics[width=0.49\textwidth]{grafik1}}
%   \hfill
%   \subfigure[Beschriftung 2\label{fig:label2}]
%   {\includegraphics[width=0.49\textwidth]{grafik2}}
%   \caption{Beschriftung allgemein}
%   \label{fig:label-gesamt}
% \end{figure}
\usepackage{subfigure}
\usepackage{adjustbox}

% Caption neben Abbildung
% Beispiel:
% \sidecaptionvpos{figure}{"c" oder "t" oder "b"}
% \begin{SCfigure}[rel. Breite (normalerweise = 1)][hbt]
%   \centering
%   \includegraphics[width=0.5\textwidth]{grafik.png}
%   \caption{Beschreibung}
%   \label{fig:}
% \end{SCfigure}
\usepackage{sidecap}

% Befehl für "Entspricht"-Zeichen
\newcommand{\corresponds}{\ensuremath{\mathrel{\widehat{=}}}}

%Für chemische Formeln (von www.dante.de)
%% Anpassung an LaTeX(2e) von Bernd Raichle
\makeatletter
\DeclareRobustCommand{\chemical}[1]{%
  {\(\m@th
   \edef\resetfontdimens{\noexpand\)%
       \fontdimen16\textfont2=\the\fontdimen16\textfont2
       \fontdimen17\textfont2=\the\fontdimen17\textfont2\relax}%
   \fontdimen16\textfont2=2.7pt \fontdimen17\textfont2=2.7pt
   \mathrm{#1}%
   \resetfontdimens}}
\makeatother

%Si Einheiten
\usepackage{siunitx}

%c++ Code einbinden
\usepackage{listings}
\lstset{numbers=left, numberstyle=\tiny, numbersep=5pt}

%Differential
\newcommand{\dif}{\ensuremath{\mathrm{d}}}
%sinc
\newcommand{\sinc}{\ensuremath{\mathrm{sinc}}}

%Boxen,etc.
\usepackage{fancybox}
\usepackage{empheq}

%Fußnoten auf gleiche Seite
\interfootnotelinepenalty=1000

%Dateien aus Unterverzeichnissen
\usepackage{import}

%Bibliography \bibliography{literatur} und \cite{gerthsen}
%\usepackage{cite}
\usepackage{babelbib}
\selectbiblanguage{ngerman}

\begin{document}

\begin{titlepage}
\centering
\textsc{\Large Anfängerpraktikum der Fakultät für
  Physik,\\[1.5ex] Universität Göttingen}

\vspace*{4.2cm}

\rule{\textwidth}{1pt}\\[0.5cm]
{\huge \bfseries
  Beugung und Interferenz\\[1.5ex]
  von Laserlicht}\\[0.5cm]
\rule{\textwidth}{1pt}

\vspace*{3.0cm}

\begin{Large}
\begin{tabular}{ll}
Praktikant:
 	&  Felix Kurtz\\
 	&  Michael Lohmann\\

E-Mail: 
	&  felix.kurtz@stud.uni-goettingen.de\\
	& m.lohmann@stud.uni-goettingen.de\\

 Betreuer: & Phillip Bastian\\
 Versuchsdatum: &  09.03.2015\\
\end{tabular}
\end{Large}

\vspace*{0.8cm}

\begin{Large}
\fbox{
  \begin{minipage}[t][2.5cm][t]{6cm} 
    Testat:
  \end{minipage}
}
\end{Large}

\end{titlepage}

\tableofcontents

\newpage

\section{Einleitung}
\label{sec:einleitung}
In diesem Versuch sollen die Eigenschaften von Laserlicht für Beugung und Interferenz an verschiedenen Objekten genutzt werden.
Da ein Laser auf stimulierter Emission von Photonen basiert, ist sein Licht nämlich sehr monochromatisch sowie zeitlich und räumlich kohärent.
Als Lichtquelle wird diesmal ein Helium-Neon-Laser verwendet, der Intensitätsverlauf wird über eine mit einem Schrittmotor bewegbare Fotodiode elektronisch aufgenommen.

\section{Theorie}
\label{sec:theorie}
\subsection{Laserprinzip}
Wie der Name \textsc{Laser}, \textit{Light Amplification by Stimulated Emission of Radiation}, schon verrät, emittiert dieser Licht aufgrund Stimulierter Emission und nicht spontan wie thermische Lichtquellen.
Dabei trifft ein Photon auf ein angeregtes Atom und es entsteht noch ein Photon, welches ununterscheidbar vom ersten ist.
Das Atom befindet sich danach in einem energieärmeren Niveau.
Für diesen Prozess benötigt man zuerst eine \emph{Besetzungsinversion} der Niveaus.
So müssen sich mehr Atome in einem höheren Energieniveau, dem oberen Laserniveau, befinden als im unteren Laserniveau.
Um eine hohe Verstärkung zu erreichen, wird meist ein \emph{Resonator} verwendet.
Dieser besteht aus einem Spiegel und einem teildurchlässigen Reflektor, dessen Transmitivität jedoch gering ist.\\
Der in diesem Versuch verwendete Laser ist ein Helium-Neon-Laser.
Er besteht aus 5/6 Helium und 1/6 Neon.
Wie in Abbildung \ref{fig:HeNe} zu sehen, werden die Heliumatome durch Elektronenstöße angeregt und geben dann ihre Energie durch Stöße an die Neonatome ab.
Dann gibt Neon Energie durch stimulierte Emission ab, meist bei $\lambda=632.8\,$nm.
Das Neonatom befindet sich nach mehreren Prozessen wieder im Grundniveau und das Ganze beginnt von vorne.

\begin{figure}[!h]
	\centering
	\includegraphics[scale=0.5]{NiveauHeNe.png}
	\caption{Niveauschema des Helium-Neon-Lasers. \cite[Datum: 02.01.15]{LP21}}
	\label{fig:HeNe}
\end{figure}


\subsection{Beugung und Interferenz}

\subsubsection{Vom Kirchhoffschen Beugungsintegral zur Fraunhofer-Näherung}
Um das Feld eines bestrahlten Objektes in einer Ebene dahinter zu berechnen, integriert man über alle \emph{Elementarwellen} der Apertur \cite[S.67]{optik}:
\begin{align}
	E(x',y',z)=\frac{1}{i \lambda} \int \limits _{Apertur} E(x,y,0) ~ \frac{e^{ikr}}{r}~\dif x ~\dif y \quad \text{mit} ~ r^2=(x'-x)^2+(y'-y)^2+z^2 \,.
	\label{eq:Beugungsintegral}
\end{align}
Dies ist das \emph{Fresnel-Kirchhoffschen Beugungsintegral}.\\

Die \emph{Fresnel-Näherung} setzt voraus, dass die Bildebene weit entfernt von der Aperturebene ist, also $z \gg x'-x,~y'-y~$ \cite[S.68]{optik}.
Dann vereinfacht sich der obige Ausdruck zu
\begin{align}
	E(x',y',z)=\frac{e^{ikz}}{i \lambda z} \int \limits _{Apertur} E(x,y,0) ~ \exp\left(\frac{ik}{2z}\left((x'-x)^2+(y'-y)^2\right)\right) ~\dif x ~\dif y\,.
	\label{eq:Fresnel}
\end{align}

Befindet man sich mit der Bildebene im \emph{Fernfeld}, kann man die \emph{Fraunhofer-Näherung} benutzen.
Dabei ist die Fresnel-Zahl $F=\frac{a^2}{\lambda z}$ mit der die Apertur charakterisierenden Größe $a$ sehr viel geringer als 1 \cite[S.68f.]{optik}.
Also können die in x bzw. y quadratischen Terme vernachlässigt werden und man erhält

\begin{align}
	E(x',y',z)=\frac{e^{ikz} \cdot e^{i \frac{k_xx'+k_yy'}{2}}}{i \lambda z}\cdot \int \limits _{Apertur} E(x,y,0)~ e^{-ik_xx} ~ e^{-ik_yy} ~\dif x~ \dif y \,,
	\label{eq:Fraunhofer1}
\end{align}
mit $k_x:= k\frac{x'}{z}$ und  $k_y:= k\frac{y'}{z}$.\\

Das Fernfeld eines bestrahlten Objekts ergibt sich also als Fouriertransformation des Feldes in der Objektebene:
\begin{align}
	E(x',y',z)=B(x,y)\cdot \mathcal{F}\left[E(x,y,0)
	\right](k_x,k_y) \,.
	\label{eq:Fraunhofer2}
\end{align}

\subsubsection{Doppelspalt}
Für einen Doppelspalt mit einem Spaltabstand $d$ und infinitesimaler Spaltbreite ergibt sich nach \eqref{eq:Fraunhofer2} dieser Intensitätsverlauf (vgl. \cite[S.457ff.]{hecht})
\begin{align}
	I(\varepsilon)=I_0 \cdot \cos^2(\varepsilon) \qquad \text{mit }\varepsilon= \frac{\pi d \sin \alpha}{\lambda}\,.
\end{align}
Man transformiert nämlich zwei Delta-Peaks.

\subsubsection{Einzelspalt und Steg}
Für einen Einzelspalt mit einer Spaltbreite $b$ ergibt sich nach \eqref{eq:Fraunhofer2} dieser Intensitätsverlauf nach \cite[S.452]{hecht}
\begin{align}
	I(\varepsilon)=I_0 \cdot \sinc^2(\varepsilon)\qquad \text{mit }\varepsilon= \frac{\pi b \sin \alpha}{\lambda}\,.
\end{align}
Da das Fernfeld eines unendlichen Schirmes, also die Superposition aus Einzelspalt und Steg, perfekt dunkel ist, gilt:
$$ E_\text{Spalt} + E_\text{Steg} = 0 \,.$$
Da man nur die Intensität, also das Quadrat des E-Feldes, misst ergibt sich als Fernfeld des Stegs der Länge $b$ die gleiche Verteilung wie bei einem Einzelspalt gleicher Spaltbreite.

\subsubsection{Kreisblende}
Für eine Kreisblende mit dem Durchmesser $D$ ergibt sich -- nach etwas komplexerer Rechnung \cite[S.467ff.]{hecht} -- nach \eqref{eq:Fraunhofer2} dieser Intensitätsverlauf
\begin{align}
	I(\varepsilon)=I_0 \cdot \left(\frac{J_1(\varepsilon)}{\varepsilon}\right)^2 \qquad \text{mit }\varepsilon= \frac{\pi D \sin \alpha}{\lambda}\,.
\end{align}
Dabei ist $J_1$ die zylindrische Besselfunktion.
Das erste Minimum von $\left(\frac{J_1(\varepsilon)}{\varepsilon}\right)^2$ liegt bei 1.22.
Diese Zahl taucht auch beim Auflösungsvermögen bzw. Rayleigh-Kriterium auf.

\subsubsection{Mehrfachspalt}
Bei einem Mehrfachspalt, bei dem $N$ Spalte der Breite $b$ und dem Abstand $d$ beleuchtet werden, erhält man die Fernfeldverteilung aus dem Faltungssatz als Produkt aus Einzelpalteffekt und der Fouriertransformation von $N$ Delta-Peaks nach \cite[S.460f.]{hecht}:
\begin{align}
	I(\varepsilon)=I_0 \cdot \sinc^2\left(\frac{\pi\alpha b}{\lambda}\right) \cdot \left(\frac{\sin(N\varepsilon)}{\sin(\varepsilon)}\right)^2 \qquad \text{mit }\varepsilon= \frac{\pi d \sin \alpha}{\lambda}\,.
\end{align}

\section{Durchführung}
\label{sec:durchfuehrung}
\begin{figure}[!h]
	\centering
	\includegraphics[scale=0.7]{Aufbau.png}
	\caption{Aufbau. \cite[Datum: 02.01.15]{LP21}}
\end{figure}

Um die Messergebnisse nicht zu verfälschen ist es immens wichtig, die Beugungsobjekte nur an der Fassung anzufassen.
Genauso vorsichtig soll auch mit den Linsen umgegangen werden.\\

Zuerst schaltet man den PC an und startet das Programm \emph{Lasersteuerung} auf dem Desktop.
Der Motor der Fotodiode bewegt sich dann auf die Nullposition.
Nun kann der Laser mittels Schiebeschalter auf der Rückseite angeschaltet werden.\\

Vor jeder Messung weitet man das Strahlenbündel mit Zerstreungs- und Sammellinse auf, sodass das gesamte Objekt homogen, parallel und ausreichend breit beleuchtet wird.
Außerdem überlegt man sich, welche Beugungsmuster zu erwarten sind.\\

Nun beginnt die Messung.
Um eine möglichst parallele Bestrahlung zu ermöglichen, muss der Abstand zwischen Objekt und Laserdiode möglichst groß sein.
Außerdem soll er gemessen werden.
Zuerst stellt man im Programm ein, von wo bis wo die Diode den Verlauf aufnehmen soll ($400~$Schritte $\corresponds 1~$mm) und welcher Spannungsbereich erfasst werden soll.
Dabei kann man je nach Intensität zwischen $0-1~$V und $0-10~$V wählen.

Die symmetrischen Spektren sollen bis zum vierten Nebenmaximum -- links wie rechts -- vom Hauptmaximum aufgenommen werden.
Außerdem genügt es beim Gitter sowie bei der Doppellochblende, nur bis zum ersten Nebenmaximum des Einzelspalteffektes bzw. der Blendenfunktion zu messen, da die Intensität danach zu gering ist.
Außerdem sollen alle Extrema deutlich bestimmt werden.
Dazu ist es wichtig den richtigen Messbereich auszuwählen, da am Rand die Maxima sehr schwach sind.
Also werden zwei Messungen durchgeführt: einmal das Zentrum und danach die Randgebiete.
In welchem Messbereich man sich befindet, kann man an der Anzeige der aktuellen Fotospannung ablesen.\\

Die Fotodiode wird mit der  manuellen Steuerung im oberen Teil des Programmfensters zuerst an die beiden Ränder der bevorstehenden Messung gefahren, die Positionswerte werden notiert und dann als Grenzen der Messung eingeben.
Nun muss noch die Schrittweite eingegeben werden.
Sie sollte unter 50 Schritten, also einem Achtel Millimeter liegen.
Im Fenster unten links wird dann der automatische Messvorgang gestartet.
Das Spektrum wird abgespeichert und mit einem Plotprogramm dargestellt, beschriftet und schließlich ausgeund meldet sich unter dem Benutzer ''laser'' ohne Passwort andruckt.
Dies dient als Messprotokoll.
Neben dem Beugungsobjekt und den Achsen mit Einheiten sollen noch der Konversionsfaktor in Schritten/mm, der Objektabstand von der Diode sowie die Wellenlänge des Lasers als auch die charakteristische Größe des Objekts enthalten.\\

Schließlich werden alle Daten auf einem USB-Stick gespeichert.

\section{Auswertung}
\label{sec:auswertung}
\subsection{Objektgrößen bestimmen}
Die Abbildungen \ref{fig:spalt}-\ref{fig:gitter} zeigen die Messdaten aller Objekte.
Daran sind im folgenden nur die Extrema interessant.

Alle erkennbaren Extrema (absolute und relative Position sowie Winkel $\alpha$) mit zugehörigem $\frac{\varepsilon}{\pi}$ sind in Tabelle \ref{tab:extrema1} und \ref{tab:extrema2} zu finden.
Dabei wird der Winkel mithilfe der Kleinwinkelnäherung berechnet: $\alpha\approx\sin\alpha=\arctan\frac{x}{l}\approx\frac{x}{l}$.
Hier bezeichnet $x$ die relative Position zum 0. Hauptmaximum und $l$ den Abstand zwischen Beugungsobjekt und Schirm.
Er beträgt $l=1.225\,$m.
Die Extremapositionen wurden durch genaue Analyse der Messdaten (Augenmaß) bestimmt.
Dabei nehmen wir eine Ungenauigkeit von $0.25\,$mm an.
Dies entspricht einer Winkelungenauigkeit von $\sigma_\alpha=0.2\cdot 10^{-3}\,$rad.
Der Wert für $\frac{\varepsilon}{\pi}$ kann aus den theoretischen Intensitätsverläufen berechnet werden, indem man die Extrema von $I(\varepsilon)$ untersucht.

In den Abbildungen \ref{fig:spalt_reg}-\ref{fig:gitter_reg} sind die Winkel $\alpha_i$ gegen den zugehörigen Wert $\frac{\varepsilon_i}{\pi}$ für die verschiedenen Objekte aufgetragen.
In allen Fällen ergibt sich eine Gerade.
Aus deren Steigung kann man nun die Objektgröße berechnen:
\begin{align}
	D&=\frac{\frac{\varepsilon}{\pi}}{\alpha}\lambda=\frac{\lambda}{m}\,,
	\label{eq:objekt}\\
	\sigma_D&=\frac{\lambda}{m^2}\cdot \sigma_m\,.
\end{align}

Zur Bestimmung der kleinsten Objektstruktur, also des Lochdurchmessers der Doppellochblenden und der Spaltbreite des Gitters, konnten nur die ersten Nullstellen der Einhüllenden genutzt werden.
Deshalb wird der Fehler auch doppelt so groß wie zuvor angenommen: $\sigma_x=0.5\,$mm.
Es ergaben sich folgende relative Positionen für die Nullstellen der Einhüllenden: Doppeloch (nah) $\pm 5\,$mm, Doppelloch (fern) $\pm 4.75\,$mm , Gitter $\pm 4\,$mm.
Wie oben auch kann man den Lochdurchmesser und die Spaltbreite berechnen.

Alle so berechneten Werte finden sich in Tabelle \ref{tab:objektgroessen} wieder.
Dabei bezeichnet $B$ die Spalt-/Stegbreite sowie den Lochabstand der Doppellochblenden, $D$ den Lochdurchmesser und $a$ den Spaltabstand.
Die berechneten Werte werden mit den Werten aus \cite[S.196]{prakti} verglichen.
Unser Versuchsaufbau war der an der Wand.
Zusätzlich vergleichen wir sie noch mit den am nächsten liegenden Werten aller drei Versuchsaufbauten.
Bei Objekten mit zwei zu bestimmenden Größen ist immer die größere für den zweiten Vergleich ausschlaggebend.

\begin{table}[!htb]
	\centering
	\begin{tabular}{|c|c||c||c|c||c|c|}
		\hline
		 & & & tatsächliche &  & beste &  \\
		Objekt & Angabe & Größe [$\mu$m] & Größe [$\mu$m] & Abw. & Größe [$\mu$m] & Abw. \\
		\hline
		\hline
		Spalt & $B$ & $216.1 \pm 0.8$ & 250 & $14\%$ & 225 & $4\%$ \\
		Steg & $B$ & $192.2 \pm 0.6$ & 200 & $4\%$ & 195 & $2\%$ \\
		Loch & $D$ & $191.1 \pm 0.4$ & 200 & $5\%$ & - & - \\
		\hline
		Doppelloch & $B$ & $486 \pm 8$ & 500 & $3\%$ & 490 & $1\%$ \\
		(nah) & $D$ & $189 \pm 19$ & 200 & $6\%$ & 210 & $10\%$ \\
		\hline
		Doppelloch & $B$& $930 \pm 11$ & 700 & $33\%$ & 1000 & $7\%$ \\
		(mittel) & $D$& $199 \pm 21$ & 200 & $1\%$ & 210 & $6\%$ \\
		\hline		
		Gitter & $a$ & $483  \pm 11$ & 285 & $70\%$ & 490 & $2\%$ \\
		& $B$ & $194 \pm 25$ & 175 & $11\%$ & 210 & $8\%$ \\
		\hline
	\end{tabular}
	\caption{Aus den Geradensteigungen berechnete Objektgrößen verglichen mit den tatsächlichen Werten}
	\label{tab:objektgroessen}
\end{table}

\begin{figure}[!htb]
	\centering
	\input{spalt_reg.tex}
	\caption{Spalt}
	\label{fig:spalt_reg}
\end{figure}

\begin{figure}[!htb]
	\centering
	\input{steg_reg.tex}
	\caption{Steg}
	\label{fig:steg_reg}
\end{figure}

\begin{figure}[!htb]
	\centering
	\input{loch_reg.tex}
	\caption{Lochblende}
	\label{fig:loch_reg}
\end{figure}

\begin{figure}[!htb]
	\centering
	% GNUPLOT: LaTeX picture with Postscript
\begingroup
  \makeatletter
  \providecommand\color[2][]{%
    \GenericError{(gnuplot) \space\space\space\@spaces}{%
      Package color not loaded in conjunction with
      terminal option `colourtext'%
    }{See the gnuplot documentation for explanation.%
    }{Either use 'blacktext' in gnuplot or load the package
      color.sty in LaTeX.}%
    \renewcommand\color[2][]{}%
  }%
  \providecommand\includegraphics[2][]{%
    \GenericError{(gnuplot) \space\space\space\@spaces}{%
      Package graphicx or graphics not loaded%
    }{See the gnuplot documentation for explanation.%
    }{The gnuplot epslatex terminal needs graphicx.sty or graphics.sty.}%
    \renewcommand\includegraphics[2][]{}%
  }%
  \providecommand\rotatebox[2]{#2}%
  \@ifundefined{ifGPcolor}{%
    \newif\ifGPcolor
    \GPcolortrue
  }{}%
  \@ifundefined{ifGPblacktext}{%
    \newif\ifGPblacktext
    \GPblacktexttrue
  }{}%
  % define a \g@addto@macro without @ in the name:
  \let\gplgaddtomacro\g@addto@macro
  % define empty templates for all commands taking text:
  \gdef\gplbacktext{}%
  \gdef\gplfronttext{}%
  \makeatother
  \ifGPblacktext
    % no textcolor at all
    \def\colorrgb#1{}%
    \def\colorgray#1{}%
  \else
    % gray or color?
    \ifGPcolor
      \def\colorrgb#1{\color[rgb]{#1}}%
      \def\colorgray#1{\color[gray]{#1}}%
      \expandafter\def\csname LTw\endcsname{\color{white}}%
      \expandafter\def\csname LTb\endcsname{\color{black}}%
      \expandafter\def\csname LTa\endcsname{\color{black}}%
      \expandafter\def\csname LT0\endcsname{\color[rgb]{1,0,0}}%
      \expandafter\def\csname LT1\endcsname{\color[rgb]{0,1,0}}%
      \expandafter\def\csname LT2\endcsname{\color[rgb]{0,0,1}}%
      \expandafter\def\csname LT3\endcsname{\color[rgb]{1,0,1}}%
      \expandafter\def\csname LT4\endcsname{\color[rgb]{0,1,1}}%
      \expandafter\def\csname LT5\endcsname{\color[rgb]{1,1,0}}%
      \expandafter\def\csname LT6\endcsname{\color[rgb]{0,0,0}}%
      \expandafter\def\csname LT7\endcsname{\color[rgb]{1,0.3,0}}%
      \expandafter\def\csname LT8\endcsname{\color[rgb]{0.5,0.5,0.5}}%
    \else
      % gray
      \def\colorrgb#1{\color{black}}%
      \def\colorgray#1{\color[gray]{#1}}%
      \expandafter\def\csname LTw\endcsname{\color{white}}%
      \expandafter\def\csname LTb\endcsname{\color{black}}%
      \expandafter\def\csname LTa\endcsname{\color{black}}%
      \expandafter\def\csname LT0\endcsname{\color{black}}%
      \expandafter\def\csname LT1\endcsname{\color{black}}%
      \expandafter\def\csname LT2\endcsname{\color{black}}%
      \expandafter\def\csname LT3\endcsname{\color{black}}%
      \expandafter\def\csname LT4\endcsname{\color{black}}%
      \expandafter\def\csname LT5\endcsname{\color{black}}%
      \expandafter\def\csname LT6\endcsname{\color{black}}%
      \expandafter\def\csname LT7\endcsname{\color{black}}%
      \expandafter\def\csname LT8\endcsname{\color{black}}%
    \fi
  \fi
  \setlength{\unitlength}{0.0500bp}%
  \begin{picture}(7200.00,5040.00)%
    \gplgaddtomacro\gplbacktext{%
      \csname LTb\endcsname%
      \put(990,704){\makebox(0,0)[r]{\strut{}-3}}%
      \put(990,1383){\makebox(0,0)[r]{\strut{}-2}}%
      \put(990,2061){\makebox(0,0)[r]{\strut{}-1}}%
      \put(990,2740){\makebox(0,0)[r]{\strut{} 0}}%
      \put(990,3419){\makebox(0,0)[r]{\strut{} 1}}%
      \put(990,4097){\makebox(0,0)[r]{\strut{} 2}}%
      \put(990,4776){\makebox(0,0)[r]{\strut{} 3}}%
      \put(1122,484){\makebox(0,0){\strut{}-2}}%
      \put(1852,484){\makebox(0,0){\strut{}-1.5}}%
      \put(2581,484){\makebox(0,0){\strut{}-1}}%
      \put(3311,484){\makebox(0,0){\strut{}-0.5}}%
      \put(4040,484){\makebox(0,0){\strut{} 0}}%
      \put(4770,484){\makebox(0,0){\strut{} 0.5}}%
      \put(5499,484){\makebox(0,0){\strut{} 1}}%
      \put(6229,484){\makebox(0,0){\strut{} 1.5}}%
      \put(6958,484){\makebox(0,0){\strut{} 2}}%
      \put(484,2740){\rotatebox{90}{\makebox(0,0){\strut{}$\alpha_i$ [$10^{-3}~$rad]}}}%
      \put(4040,154){\makebox(0,0){\strut{}$\frac{\varepsilon_i}{\pi}$ [rad]}}%
      \put(1852,4097){\makebox(0,0)[l]{\strut{}$m = (1.3027 \pm 0.0195)~10^{-3}$}}%
      \put(1852,3690){\makebox(0,0)[l]{\strut{}$b = (0.0102 \pm 0.0251)~10^{-3}$}}%
    }%
    \gplgaddtomacro\gplfronttext{%
      \csname LTb\endcsname%
      \put(5971,1097){\makebox(0,0)[r]{\strut{}Extrema}}%
      \csname LTb\endcsname%
      \put(5971,877){\makebox(0,0)[r]{\strut{}$\alpha=m \cdot \frac{\varepsilon}{\pi}+b $}}%
    }%
    \gplbacktext
    \put(0,0){\includegraphics{doppelloch1_reg}}%
    \gplfronttext
  \end{picture}%
\endgroup

	\caption{Doppelloch-Blende mit kleinstem Lochabstand}
	\label{fig:doppelloch1_reg}
\end{figure}

\begin{figure}[!htb]
	\centering
	\input{doppelloch2_reg.tex}
	\caption{Doppelloch-Blende mit mittlerem Lochabstand}
	\label{fig:doppelloch2_reg}
\end{figure}

\begin{figure}[!htb]
	\centering
	\input{gitter_reg.tex}
	\caption{Gitter}
	\label{fig:gitter_reg}
\end{figure}

\subsection{Wellenlänge des Lasers bestimmen}
Stellt man \eqref{eq:objekt} nach der Wellenlänge $\lambda$ um, erhält man folgende Formel und ihre Fehlerabschätzung:
\begin{align}
	\lambda&=mD\,,\\
	\sigma_\lambda&=D\cdot\sigma_m\,.
\end{align}
Den besten Korrelationskoeffizienten wies die \textbf{Lochblenden-Messung} mit $r=0.99998$ auf.
Setzt man einen Durchmesser $D=200\,\mu$m der Lochblende voraus und verwendet die in Abbildung \ref{fig:loch_reg} gefittete Geradensteigung $m$ ergibt sich folgende Wellenlänge des HeNe-Lasers:
\begin{empheq}[box=\shadowbox]{align*}
	\lambda=(662.4\pm 1.4)\,\si{\nano\meter}\,.
\end{empheq}
Dieser Wert ist etwa $5\%$ größer als der Literaturwert $\lambda=632.8\,$nm.


\section{Diskussion}
\label{sec:diskussion}
\subsection{Objektgrößen}
Vergleicht man die Messwerte mit den tatsächlichen Größen wie in Tabelle \ref{tab:objektgroessen} fallen einem sehr große Abweichungen wie zB.: beim Gitter von $70\%$ oder beim Doppelloch (mittel) von $33\%$ auf.
Da dies eine Systematik zu haben scheint, werden die Werte auch noch mit den Werten der anderen Versuchsaufbauten verglichen.
In allen Fällen ist der nächste Wert der eines Objekts vom mittleren Versuchsaufbau und die Fehler liegen unter $10\%$.
Bei den Objekten Spalt, Steg und der Lochblende sowie der dem Doppelloch mit mittlerem Lochabstand wurden die Fehler jedoch zu klein abgeschätzt, da auch die besseren Werte nicht in den Fehlerintervallen liegen.
Für die kleinen Strukturen wurde ein großer Fehler angenommen, da es nur zwei Messwerte gab.
Alle drei Aufbauten haben ähnliche Gößen für diese Strukturen, sodass sich daraus nicht verlässlich sagen lässt, welches Objekt gemessen wurde.
Dies kann man jedoch an den großen Strukturen gut ablesen.\\

Somit nehmen wir an, dass die Sets der Beugungsobjekte vom mittleren Versuchsaufbau sich bei uns an der Wand befanden und umgekehrt.
Man sollte nochmal nachmessen und alle Praktikanten darauf hinweisen, dass die drei Versuchsaufbauten nicht die gleichen Objektgrößen benutzen und man die Objekte nicht vertauschen sollte.
Zusätzlich könnte man sie markieren.
Dies alles lässt eine bessere Vergleichbarkeit der Messwerte mit den angegebenen Werten zu.

\subsection{Wellenlänge}
Die Wellenlänge des Lasers ist $5\%$ kleiner als berechnet und liegt auch nicht im Fehlerintervall.
Da aber die Werte in Abb.\ref{fig:loch_reg} einen hohen Korrelationskoeffizienten aufweisen, kann der Fehler nur bei der Größe der Lochblende gesucht werden.
Laut der Tabelle auf Seite 196 aus \cite{prakti} ist sie jedoch für alle Objekte gleich groß.
Sie kann also auch nicht aus einem anderen Versuchsaufbau stammen.
Es könnte sich aber Schmutz abgelagert haben, sodass der Durchmesser geringer ist als angenommen.
Dann wird die berechnete Laserwellenlänge geringer.
Man sollte den Durchmesser also nochmal nachmessen.
Außerdem hätte man für ihn einen Fehler annehmen sollen.
Da wir allerdings nicht wissen, wie und vor allem wie genau die Bestimmung des Durchmessers verlief, können wir keinen vernünftig begründbaren Fehler angeben.

\section{Anhang}
\subsection{Messwerte}
Abbildungen \ref{fig:spalt}-\ref{fig:gitter}
\begin{figure}[!htb]
	\centering
	\input{spalt.tex}
	\caption{Einzelspalt}
	\label{fig:spalt}
\end{figure}

\begin{figure}[!htb]
	\centering
	\input{spalt_grob.tex}
	\caption{Einzelspalt grob}
\end{figure}

\begin{figure}[!htb]
	\centering
	\input{steg_mitte.tex}
	\caption{Steg: Hauptmaximum}
\end{figure}

\begin{figure}[!htb]
	\centering
	\input{steg.tex}
	\caption{Steg}
\end{figure}

\begin{figure}[!htb]
	\centering
	\input{loch.tex}
	\caption{Lochblende}
\end{figure}

\begin{figure}[!htb]
	\centering
	\input{doppelloch1.tex}
	\caption{Doppellochblende mit dem kleinsten Lochabstand}
\end{figure}

\begin{figure}[!htb]
	\centering
	\input{doppelloch2.tex}
	\caption{Doppellochblende mit dem mittleren Lochabstand}
\end{figure}

\begin{figure}[!htb]
	\centering
	\input{gitter.tex}
	\caption{Gitter}
	\label{fig:gitter}
\end{figure}

\subsection{Extrema-Tabellen}
Tabellen \ref{tab:extrema1} und \ref{tab:extrema2}
\begin{table}[!htb]
	\centering
	\begin{tabular}{|c||c|c|c|c|}
		\hline		
		Objekt & $\frac{\varepsilon}{\pi}$ & \multicolumn{2}{c}{Position [mm]} & Winkel $\alpha$ \\
		& & absolut & relativ & [$10^{-3}~$rad] \\
		\hline
		\hline
		&	-4	&	35.225	&	-14.575	&	-2.974	\\
		&	-3.5	&	37.288	&	-12.513	&	-2.554	\\
		&	-3	&	38.825	&	-10.975	&	-2.240	\\
		&	-2.5	&	40.850	&	-8.950	&	-1.827	\\
		&	-2	&	42.575	&	-7.225	&	-1.474	\\
		&	-1.5	&	44.575	&	-5.225	&	-1.066	\\
		&	-1	&	46.025	&	-3.775	&	-0.770	\\
		Spalt &	0	&	49.800	&	0.000	&	0.000	\\
		&	1	&	53.375	&	3.575	&	0.730	\\
		&	1.5	&	54.900	&	5.100	&	1.041	\\
		&	2	&	56.963	&	7.163	&	1.462	\\
		&	2.5	&	58.650	&	8.850	&	1.806	\\
		&	3	&	60.525	&	10.725	&	2.189	\\
		&	3.5	&	62.250	&	12.450	&	2.541	\\
		&	4	&	64.150	&	14.350	&	2.929	\\
		\hline
		&	-4	&	33.050	&	-16.200	&	-3.306	\\
		&	-3.5	&	35.225	&	-14.025	&	-2.862	\\
		&	-3	&	37.138	&	-12.113	&	-2.472	\\
		&	-2.5	&	39.238	&	-10.013	&	-2.043	\\
		&	-2	&	41.175	&	-8.075	&	-1.648	\\
		&	-1.5	&	43.475	&	-5.775	&	-1.179	\\
		&	-1	&	45.175	&	-4.075	&	-0.832	\\
		Steg &	0	&	49.250	&	0.000	&	0.000	\\
		&	1	&	53.400	&	4.150	&	0.847	\\
		&	1.5	&	55.088	&	5.838	&	1.191	\\
		&	2	&	57.450	&	8.200	&	1.673	\\
		&	2.5	&	59.250	&	10.000	&	2.041	\\
		&	3	&	61.425	&	12.175	&	2.485	\\
		&	3.5	&	63.300	&	14.050	&	2.867	\\
		&	4	&	65.550	&	16.300	&	3.327	\\
		\hline
		&	3.6987	&	34.125	&	-15.075	&	-3.077	\\
		&	3.2383	&	36.075	&	-13.125	&	-2.679	\\
		&	2.6793	&	38.325	&	-10.875	&	-2.219	\\
		&	2.2331	&	40.125	&	-9.075	&	-1.852	\\
		&	1.6347	&	42.600	&	-6.600	&	-1.347	\\
		&	1.2197	&	44.100	&	-5.100	&	-1.041	\\
		Loch- &	0	&	49.200	&	0.000	&	0.000	\\
		blende &	1.2197	&	54.075	&	4.875	&	0.995	\\
		&	1.6347	&	55.800	&	6.600	&	1.347	\\
		&	2.2331	&	58.350	&	9.150	&	1.867	\\
		&	2.6793	&	60.150	&	10.950	&	2.235	\\
		&	3.2383	&	62.400	&	13.200	&	2.694	\\
		&	3.6987	&	64.125	&	14.925	&	3.046	\\
		&	4.2411	&	66.250	&	17.050	&	3.480	\\
		\hline
	\end{tabular}
	\caption{Spalt, Steg und Lochblende: Extrema absolut und relativ sowie zugehöriger Winkel}
	\label{tab:extrema1}
\end{table}

\begin{table}[!htb]
	\centering
	\begin{tabular}{|c||c|c|c|c|}
		\hline		
		Objekt & $\frac{\varepsilon}{\pi}$ & \multicolumn{2}{c}{Position [mm]} & Winkel $\alpha$ \\
		& & absolut & relativ & [$10^{-3}~$rad] \\
		\hline
		\hline
		&	-2	&	46.200	&	-3.088	&	-0.630	\\
		&	-1.5	&	46.800	&	-2.488	&	-0.508	\\
		&	-1	&	47.713	&	-1.575	&	-0.321	\\
		Doppel- &	-0.5	&	48.463	&	-0.825	&	-0.168	\\
		loch &	0	&	49.288	&	0.000	&	0.000	\\
		(nah) &	0.5	&	50.113	&	0.825	&	0.168	\\
		&	1	&	50.900	&	1.613	&	0.329	\\
		&	1.5	&	51.850	&	2.563	&	0.523	\\
		&	2	&	52.375	&	3.088	&	0.630	\\
		\hline
		&	-3	&	47.050	&	-2.250	&	-0.459	\\
		&	-2.5	&	47.300	&	-2.000	&	-0.408	\\
		&	-2	&	47.800	&	-1.500	&	-0.306	\\
		&	-1.5	&	48.100	&	-1.200	&	-0.245	\\
		&	-1	&	48.450	&	-0.850	&	-0.173	\\
		Doppel- &	-0.5	&	48.900	&	-0.400	&	-0.082	\\
		loch &	0	&	49.300	&	0.000	&	0.000	\\
		(mittel) &	0.5	&	49.725	&	0.425	&	0.087	\\
		&	1	&	50.150	&	0.850	&	0.173	\\
		&	1.5	&	50.700	&	1.400	&	0.286	\\
		&	2	&	51.050	&	1.750	&	0.357	\\
		&	2.5	&	51.500	&	2.200	&	0.449	\\
		&	3	&	51.900	&	2.600	&	0.531	\\
		&	3.5	&	52.350	&	3.050	&	0.622	\\
		&	4	&	52.775	&	3.475	&	0.709	\\
		\hline
		&	-4.5	&	42.250	&	-7.000	&	-1.429	\\
		&	-4	&	42.850	&	-6.400	&	-1.306	\\
		&	-3.5	&	43.575	&	-5.675	&	-1.158	\\
		&	-3	&	44.575	&	-4.675	&	-0.954	\\
		&	-2.5	&	45.000	&	-4.250	&	-0.867	\\
		&	-2	&	46.000	&	-3.250	&	-0.663	\\
		Gitter &	-1.5	&	46.400	&	-2.850	&	-0.582	\\
		&	-1	&	47.675	&	-1.575	&	-0.321	\\
		&	-0.5	&	48.250	&	-1.000	&	-0.204	\\
		&	0	&	49.250	&	0.000	&	0.000	\\
		&	0.5	&	50.025	&	0.775	&	0.158	\\
		&	1	&	50.850	&	1.600	&	0.327	\\
		&	1.5	&	53.825	&	4.575	&	0.934	\\
		&	2	&	52.000	&	2.750	&	0.561	\\
		&	2.5	&	53.825	&	4.575	&	0.934	\\
		\hline
	\end{tabular}
	\caption{Doppellochblenden und Gitter: Extrema absolut und relativ sowie zugehöriger Winkel}	
	\label{tab:extrema2}
\end{table}

\bibliography{literatur}
\bibliographystyle{babalpha}

\end{document}
